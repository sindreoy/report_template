\section{Conclusion}
\label{sec:conclusion}
The conclusion should, as the abstract, wrap up what you have found in the experiment. It should state what you have done and what you have found. The conclusion should only state what is obvious from the discussion in section \ref{sec:results}; no new information should arise here. The conclusion is the first place the reader will go looking if he wants to get an overview of the report. If it is interesting, he might read the rest. Be sure that the conclusion is short and concise, but do not omit important information. You have one shot at presenting your results: if you have done excellent work at the lab it doesn't matter if you are unable to present the results in an appealing way. The report is your only way of communicating and presenting your hard work.